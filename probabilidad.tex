\section{Probabilidad}

Imaginemos un ejemplo simple: lanzar un dado de seis caras. Este es un experimento aleatorio porque
 no podemos predecir con certeza el resultado antes de realizarlo. Pero, si sabemos cuales son los resultados posibles.
 Al tratarse de un dado de seis caras, los resultados posibles son $\{1, 2, 3, 4, 5, 6\}$, esto es el \textbf{Espacio muestral}
 de nuestro experimento y lo denotaremos como $\Omega$.

$$
\Omega = \{1, 2, 3, 4, 5, 6\}
$$

Sin embargo, para este experimento en particular, no nos interesa saber el número que salió al lanzar el dado, sino que nos interesa
saber si el número que salió es par o impar. Sabemos que los números pares que podemos obtener son $\{2, 4, 6\}$ y los impares son $\{1, 3, 5\}$.
Estos dos conjuntos son \textbf{eventos} de nuestro experimento, y los denotaremos como $A$ y $B$ respectivamente. Al tomar el conjunto de
todos nuestros eventos posibles, obtenemos el \textbf{Espacio de sucensos} de nuestro experimento, denotado por $\mathcal{F}$.

$$
\mathcal{F} = \{\{1, 3, 5\}, \{2, 4, 6\}\}
$$

Ahora, queremos asignar un valor numérico que, para cada evento de nuestro espacio de sucesos, nos indique la certidumbre de que
ese evento ocurra. A esta medida de certidumbre la llamaremos \textbf{Probabilidad}. Si nuestro dado es justo, entonces
cada evento de nuestro espacio de sucesos tiene la misma probabilidad de ocurrir. Pero, ¿cómo asignamos esta probabilidad?

Una forma sencilla sería contar cuántos resultados favorecen a cada evento. Para $A=\{2,4,6\}$, hay 3 resultados favorables.
Para $B=\{1,3,5\}$, también hay 3 resultados favorables. Si comparamos estos números con el total de resultados posibles ($6$), podemos decir
que cada evento ocurre en $3$ de $6$ casos, es decir, en $\frac{1}{2}$ de los casos. La función que asigna a cada evento su probabilidad
se llama \textbf{Función de probabilidad} y se denota como $P$.

$$
P(A) = P(B) = \frac{1}{2}
$$

Acabamos de definir las tres componente básicas de la teoría de la probabilidad: el espacio muestral, los eventos y la probabilidad. A este
conjunto de componentes lo llamamos \textbf{Espacio de probabilidad} y lo denotamos como $(\Omega, \mathcal{F}, P)$.



