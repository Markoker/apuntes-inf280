\documentclass{article}
\usepackage{amsmath, amsthm, amssymb}
\usepackage{graphicx}
\usepackage{hyperref}      % Para hipervínculos
\usepackage{geometry}
\usepackage{multicol}
\usepackage{array}
\geometry{a4paper, left=20mm, right=20mm, top=20mm, bottom=20mm}

\newcommand{\Distribucion}[6]{
  $ 
    \begin{array}{ll}
      \textbf{Distribución #1} & #2 \\
      f(x) = #3 & F(x) = #4 \\
      \mathbb{E}[X] = #5 & \text{V}[X] = #6
    \end{array}
  $
}


\title{Apuntes INF280}
\author{Marco Repetto}
\date{}

\begin{document}
\maketitle
\tableofcontents
\newpage

% Incluye cada sección en el documento principal
\section{Formulario}


\subsection{Probabilidad}
\begin{itemize}
    \item \hyperref[sec:axiomasKolmogorov]{\textbf{Axiomas de Kolmogorov}}

    \begin{enumerate}
        \item $P(A) \geq 0$ para todo evento $A$. 
        \item $P(\Omega) = 1$, donde $\Omega$ es el espacio muestral. 
        \item Si $A_1, A_2, \ldots$ son eventos mutuamente excluyentes, entonces:
        \[
          P\left(\bigcup_{i=1}^{\infty} A_i\right) = \sum_{i=1}^{\infty} P(A_i)
        \]
    \end{enumerate}

    \item \hyperref[sec:probComplemento]{\textbf{Probabilidad del complemento}}:
    $P(A^c) = 1 - P(A)$

    \item \hyperref[sec:reglaAdicion]{\textbf{Regla de la adición}}: 
    $P(A \cup B) = P(A) + P(B) - P(A \cap B)$

    \item \hyperref[sec:reglaMultiplicacion]{\textbf{Regla de la multiplicación}}:
    \begin{itemize}
      \item $P(A \cap B) = P(A) \cdot P(B)$, $A$ y $B$ independientes.
      \item $P(A \cap B) = P(A) \cdot P(B|A)$, $A$ y $B$ dependientes.
    \end{itemize}
\end{itemize}

\subsubsection{Probabilidad condicional}
\begin{itemize}
    \item \hyperref[sec:probabilidadCondicional]{\textbf{Definición}}: $$P(A|B) = \frac{P(A \cap B)}{P(B)}$$

    \item \hyperref[sec:independencia]{\textbf{Independencia}}: $$P(A|B) = P(A) \quad \text{o} \quad P(B|A) = P(B)$$

    \item \hyperref[sec:leyProbTotal]{\textbf{Ley de la probabilidad total}}: $$P(B) = \sum_{i=1}^{n} P(A_i) \cdot P(B|A_i)$$

    \item \hyperref[sec:reglaBayes]{\textbf{Regla de Bayes}}: $$P(A|B) = \frac{P(B|A) \cdot P(A)}{P(B)}$$
\end{itemize}

\subsection{Variables aleatorias}
\subsubsection{Variables aleatorias discretas}
\begin{itemize}
    \item \hyperref[sec:variableAleatoria]{\textbf{Definición}}: Una variable aleatoria es una función que asigna un número real a cada resultado en el espacio muestral.

    \item \hyperref[sec:funcionDistribucion]{\textbf{Función de distribución}}: $F(x) = P(X \leq x)$

    \item \hyperref[sec:esperanza]{\textbf{Esperanza}}: $E(X) = \sum_{i=1}^{n} x_i \cdot P(X = x_i)$
    \item \hyperref[sec:varianza]{\textbf{Varianza}}: $Var(X) = \sum_{i=1}^{n} (x_i - E(X))^2 \cdot P(X = x_i)$ 
\end{itemize}

\subsubsection{Variables aleatorias continuas}
\begin{itemize}
    \item \hyperref[sec:variableAleatoria]{\textbf{Definición}}: Una variable aleatoria es una función que asigna un número real a cada resultado en el espacio muestral.

    \item \hyperref[sec:funcionDensidad]{\textbf{Función de densidad}}:
    \begin{enumerate}
      \item $f(x) \geq 0$ para todo $x$. 
      \item $$\int_{-\infty}^{\infty} f(x) \, dx = 1$$. 
    \end{enumerate}

    \item \hyperref[sec:funcionDistribucion]{\textbf{Función de distribución}}: $F(x) = P(X \leq x) = \int_{-\infty}^{x} f(t) \, dt$

    \item \hyperref[sec:esperanza]{\textbf{Esperanza}}: $E(X) = \int_{-\infty}^{\infty} x \cdot f(x) \, dx$
    \item \hyperref[sec:varianza]{\textbf{Varianza}}: $Var(X) = \int_{-\infty}^{\infty} (x - E(X))^2 \cdot f(x) \, dx$
\end{itemize}

\subsubsection{Propiedades}
\begin{itemize}
    \item \hyperref[sec:esperanza]{\textbf{Esperanza}}: $E[XY] = E[X]E[Y]$ si $X$ y $Y$ son independientes.
    \item \hyperref[sec:varianza]{\textbf{Varianza}}: $Var(X) = E(X^2) - (E(X))^2$ 

    \item \hyperref[sec:esperanzaCombinacion]{\textbf{Esperanza de una combinación lineal}}: $E(aX + b) = a \cdot E(X) + b$

    \item \hyperref[sec:varianzaCombinacion]{\textbf{Varianza de una combinación lineal}}: $Var(aX + b) = a^2 \cdot Var(X)$
\end{itemize}

\subsection{Multivariadas}
\begin{itemize}
  \item \hyperref[sec:probabilidadMarginal]{\textbf{Probabilidad marginal}}: $P(X = x) = \int_{-\infty}^{\infty} f(x, y) \, dy = \sum_{j=1}^{m} p(x, y_j)$

  \item \hyperref[sec:independencia]{\textbf{Independencia}}: $f(x, y) = f_X(x) \cdot f_Y(y)$

  \item \hyperref[sec:probabilidadCondicionalMult]{\textbf{Probabilidad condicional}}: $P(X = x | Y = y) = \frac{P(X = x \cap Y = y)}{P(Y = y)}$ 
 
  \item \hyperref[sec:esperanzaCondicionada]{\textbf{Esperanza condicionada}}: $E(X|Y = y) = \int_{-\infty}^{\infty} x \cdot f(x|y) \, dx$

  \item \hyperref[sec:esperanzaMultivariada]{\textbf{Esperanza multivariada}}: $E(X) = \begin{bmatrix} E(X_1) \\ E(X_2) \\ \vdots \\ E(X_n) \end{bmatrix}$

  \item \hyperref[sec:varianzaMultivariada]{\textbf{Varianza multivariada}}: $Var(X) = \begin{bmatrix} Var(X_1) & Cov(X_1, X_2) & \cdots & Cov(X_1, X_n) \\ Cov(X_2, X_1) & Var(X_2) & \cdots & Cov(X_2, X_n) \\ \vdots & \vdots & \ddots & \vdots \\ Cov(X_n, X_1) & Cov(X_n, X_2) & \cdots & Var(X_n) \end{bmatrix}$

  \item \hyperref[sec:covarianza]{\textbf{Covarianza}}: 
  \begin{enumerate}
    \item $Cov(X, Y) = E[(X - E[X])(Y - E[Y])] = E[XY] - E[X]E[Y]$
    \item $Cov(X, X) = Var(X)$
    \item $Cov(X, Y) = Cov(Y, X)$ 
    \item $Cov(aX + bY, cW + dV) = ac \cdot Cov(X, W) + ad \cdot Cov(X, V) + bc \cdot Cov(Y, W) + bd \cdot Cov(Y, V)$ 
    \item Si $X$ y $Y$ son independientes, entonces $Cov(X, Y) = 0$ 
    \item $V[X+Y] = V[X] + V[Y] + 2 \cdot Cov(X, Y)$
  \end{enumerate}

  \item \hyperref[sec:correlacion]{\textbf{Correlación}}:
  \begin{enumerate}
    \item $Corr(X, Y) = \frac{Cov(X, Y)}{\sqrt{Var(X) \cdot Var(Y)}}$
    \item $-1 \leq Corr(X, Y) \leq 1$
  \end{enumerate}
\end{itemize}

\subsection{Distribuciones}
\subsubsection{Discretas}

\Distribucion{Bernoulli}{X \sim \text{Bernoulli}(p)}{p^x(1-p)^{1-x}}{-}{p}{p(1-p)}
\Distribucion{Binomial}{X \sim \text{Binomial}(n, p)}{\binom{n}{x} p^x(1-p)^{n-x}}{F(x) = \sum_{i=0}^{x} \binom{n}{i} p^i(1-p)^{n-i}}{np}{np(1-p)}
\Distribucion{Poisson}{X \sim \text{Poisson}(\lambda)}{\frac{e^{-\lambda} \lambda^x}{x!}}{F(x) = e^{-\lambda} \sum_{i=0}^{x} \frac{\lambda^i}{i!}}{\lambda}{\lambda}       
\Distribucion{Geométrica}{X \sim \text{Geométrica}(p)}{(1-p)^{x-1}p}{F(x) = 1 - (1-p)^x}{\frac{1}{p}}{\frac{1-p}{p^2}}
\Distribucion{Hipergeométrica}{X \sim \text{Hipergeométrica}(N, n, m)}{\frac{\binom{m}{x} \binom{N-m}{n-x}}{\binom{N}{n}}}{F(x) = \sum_{i=0}^{x} \frac{\binom{m}{i} \binom{N-m}{n-i}}{\binom{N}{n}}}{n \cdot \frac{m}{N}}{n \cdot \frac{m}{N} \cdot \frac{N-m}{N} \cdot \frac{N-n}{N-1}}

\subsubsection{Continuas}

\Distribucion{Uniforme}{X \sim \text{Uniforme}(a, b)}{\frac{1}{b-a}}{F(x) = \frac{x-a}{b-a}}{\frac{a+b}{2}}{\frac{(b-a)^2}{12}}
\Distribucion{Exponencial}{X \sim \text{Exponencial}(\lambda)}{\lambda e^{-\lambda x}}{F(x) = 1 - e^{-\lambda x}}{\frac{1}{\lambda}}{\frac{1}{\lambda^2}}
\Distribucion{Normal}{X \sim \text{Normal}(\mu, \sigma^2)}{\frac{1}{\sqrt{2\pi}\sigma} e^{-\frac{(x-\mu)^2}{2\sigma^2}}}{F(x) = \frac{1}{2} \left[1 + \text{erf}\left(\frac{x-\mu}{\sigma \sqrt{2}}\right)\right]}{\mu}{\sigma^2} \\
\Distribucion{Normal estándar}{Z = \frac{X - \mu}{\sigma} \sim \text{Normal}(0, 1)}{\frac{1}{\sqrt{2\pi}} e^{-\frac{x^2}{2}}}{F(x) = \frac{1}{2} \left[1 + \text{erf}\left(\frac{x}{\sqrt{2}}\right)\right]}{0}{1} \\
\Distribucion{Chi-cuadrado}{X \sim \chi^2(n)}{\frac{1}{2^{n/2} \Gamma(n/2)} x^{n/2-1} e^{-x/2}}{F(x) = \frac{1}{\Gamma(n/2)} \gamma\left(\frac{n}{2}, \frac{x}{2}\right)}{n}{2n} \\
\Distribucion{t de Student}{X \sim t(n)}{\frac{\Gamma\left(\frac{n+1}{2}\right)}{\sqrt{n\pi} \Gamma\left(\frac{n}{2}\right)} \left(1 + \frac{x^2}{n}\right)^{-\frac{n+1}{2}}}{F(x) = \frac{1}{2} + \frac{1}{2} \text{sign}(x) \left(1 - \frac{2}{\pi} \arctan\left(\sqrt{\frac{n-2}{n}} x\right)\right)}{0}{\frac{n}{n-2} \quad \text{para } n > 2} \\
\Distribucion{Gamma}{X \sim \Gamma(\alpha, \beta)}{\frac{\beta^\alpha}{\Gamma(\alpha)} x^{\alpha-1} e^{-\beta x}}{F(x) = \frac{1}{\Gamma(\alpha)} \gamma(\alpha, \beta x)}{\frac{\alpha}{\beta}}{\frac{\alpha}{\beta^2}}  
\Distribucion{Beta}{X \sim \text{Beta}(\alpha, \beta)}{\frac{\Gamma(\alpha + \beta)}{\Gamma(\alpha) \Gamma(\beta)} x^{\alpha-1} (1-x)^{\beta-1}}{F(x) = I_x(\alpha, \beta)}{\frac{\alpha}{\alpha + \beta}}{\frac{\alpha \beta}{(\alpha + \beta)^2 (\alpha + \beta + 1)}}  
\Distribucion{Weibull}{X \sim \text{Weibull}(\lambda, k)}{k \lambda x^{k-1} e^{-\lambda x^k}}{F(x) = 1 - e^{-\lambda x^k}}{\frac{1}{\lambda} \Gamma\left(1 + \frac{1}{k}\right)}{\frac{1}{\lambda^2} \left[\Gamma\left(1 + \frac{2}{k}\right) - \left(\Gamma\left(1 + \frac{1}{k}\right)\right)^2\right]}


\subsubsection{Multivariadas}

\Distribucion{Normal multivariada}{X \sim \text{Normal}(\boldsymbol{\mu}, \boldsymbol{\Sigma})}{\frac{1}{(2\pi)^{n/2} |\boldsymbol{\Sigma}|^{1/2}} e^{-\frac{1}{2} (\boldsymbol{x} - \boldsymbol{\mu})^T \boldsymbol{\Sigma}^{-1} (\boldsymbol{x} - \boldsymbol{\mu})}}{-}{\boldsymbol{\mu}}{\boldsymbol{\Sigma}}\\
\Distribucion{Multinomial}{X \sim \text{Multinomial}(n, \boldsymbol{p})}{\frac{n!}{x_1! \cdots x_k!} p_1^{x_1} \cdots p_k^{x_k}}{-}{n \cdot p_1}{n \cdot p_1(1-p_1) \cdots (1-p_k)}   

\section{Utilidades}

\begin{itemize}
  \item Calculadora de probabilidades distribución normal estándar: \url{https://homepage.divms.uiowa.edu/~mbognar/applets/normal.html}
  \item Galería de gráficos en Python: \url{https://python-graph-gallery.com/}
  \item Galería de gráficos en R: \url{https://www.r-graph-gallery.com/}
\end{itemize}

% \section{Probabilidad}

Imaginemos un ejemplo simple: lanzar un dado de seis caras. Este es un experimento aleatorio porque
 no podemos predecir con certeza el resultado antes de realizarlo. Pero, si sabemos cuales son los resultados posibles.
 Al tratarse de un dado de seis caras, los resultados posibles son $\{1, 2, 3, 4, 5, 6\}$, esto es el \textbf{Espacio muestral}
 de nuestro experimento y lo denotaremos como $\Omega$.

$$
\Omega = \{1, 2, 3, 4, 5, 6\}
$$

Sin embargo, para este experimento en particular, no nos interesa saber el número que salió al lanzar el dado, sino que nos interesa
saber si el número que salió es par o impar. Sabemos que los números pares que podemos obtener son $\{2, 4, 6\}$ y los impares son $\{1, 3, 5\}$.
Estos dos conjuntos son \textbf{eventos} de nuestro experimento, y los denotaremos como $A$ y $B$ respectivamente. Al tomar el conjunto de
todos nuestros eventos posibles, obtenemos el \textbf{Espacio de sucensos} de nuestro experimento, denotado por $\mathcal{F}$.

$$
\mathcal{F} = \{\{1, 3, 5\}, \{2, 4, 6\}\}
$$

Ahora, queremos asignar un valor numérico que, para cada evento de nuestro espacio de sucesos, nos indique la certidumbre de que
ese evento ocurra. A esta medida de certidumbre la llamaremos \textbf{Probabilidad}. Si nuestro dado es justo, entonces
cada evento de nuestro espacio de sucesos tiene la misma probabilidad de ocurrir. Pero, ¿cómo asignamos esta probabilidad?

Una forma sencilla sería contar cuántos resultados favorecen a cada evento. Para $A=\{2,4,6\}$, hay 3 resultados favorables.
Para $B=\{1,3,5\}$, también hay 3 resultados favorables. Si comparamos estos números con el total de resultados posibles ($6$), podemos decir
que cada evento ocurre en $3$ de $6$ casos, es decir, en $\frac{1}{2}$ de los casos. La función que asigna a cada evento su probabilidad
se llama \textbf{Función de probabilidad} y se denota como $P$.

$$
P(A) = P(B) = \frac{1}{2}
$$

Acabamos de definir las tres componente básicas de la teoría de la probabilidad: el espacio muestral, los eventos y la probabilidad. A este
conjunto de componentes lo llamamos \textbf{Espacio de probabilidad} y lo denotamos como $(\Omega, \mathcal{F}, P)$.




% \include{variables_aleatorias}
% \include{estadistica_descriptiva}
% \include{inferencia}

\end{document}

