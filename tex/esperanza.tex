
\subsection{Esperanza}

Volvamos al juego propuesto por nuestro amigo, recordemos que planteamos lo siguiente en base a nuestro dado:

$$
X(\omega) = \left\{
    \begin{array}{ll}
        -200 & \text{si } x \leq 4 \\
        300 & \text{si } x > 4
    \end{array}
\right.
$$

Ya que somos unos apostadores responsables, antes de aceptar la propuesta nos interesa saber si nos conviene jugar o no.
A simple vista, no podemos determinar si el juego es favorable o no, por lo tanto para saber si el juego vale la pena,
necesitamos una forma de resumir estos posibles resultados en un solo número que represente lo que esperamos ganar o
perder en promedio.

Comparemos los resultados posibles del juego con la cantidad de dinero que ganamos o perdemos.
Ya que sabemos que los resultados de nuestro dado son equiprobables sabemos que,

\begin{align*}
    P(X = -200) &= P(X \leq 4) \\
    &= P(X = 1) + P(X = 2) + P(X = 3) + P(X = 4) \\
    &= \frac{1}{6} + \frac{1}{6} + \frac{1}{6} + \frac{1}{6} \\
    &= \frac{4}{6} = \frac{2}{3} \\
    P(X = 300) &= P(X > 4) \\
    &= P(X = 5) + P(X = 6) \\
    &= \frac{1}{6} + \frac{1}{6}\\
    &= \frac{2}{6} = \frac{1}{3}
\end{align*}

Por lo tanto podemos decir de manera intuitiva que lo esperado es que en $\frac{2}{3}$ de los casos perderemos 200 pesos y en $\frac{1}{3}$
ganaremos 300 pesos. Con esto si calculamos nuestra \textit{ganancia esperada} obtenemos,

$$
\mathbb{E}[X] = -200 \cdot \frac{2}{3} + 300 \cdot \frac{1}{3} = -33.3
$$

Esto es lo que se conoce como \textbf{esperanza} de la variable aleatoria $X$, la cual  representa el valor promedio que
obtendríamos si jugáramos este juego muchas veces. No es lo que ocurrirá en una sola tirada, pero sí lo que podríamos
anticipar a largo plazo. En este caso, la esperanza es negativa, lo que indica que el juego no es favorable para nosotros.

\begin{definicion}{Esperanza (discreta)}
    Sea $X$ una variable aleatoria discreta. La \textbf{esperanza} de $X$, denotada por $\mathbb{E}[X]$, es el valor
    promedio que se espera obtener al realizar el experimento aleatorio muchas veces. Se define como:
    $$
        \mathbb{E}[X] = \sum_{x \in R_X} x \cdot P(X = x)
    $$
    donde $R_X$ es el rango de la variable aleatoria $X$.
\end{definicion}

Ahora obtendremos algunas propiedades utiles de la esperanza, las cuales nos ayudaran a resolver problemas.

\begin{definicion}{Propiedades de la esperanza}
    Sea $X$ y $Y$ variables aleatorias discretas y $c \in \mathbb{R}$. Entonces se cumplen las siguientes propiedades:
    \begin{itemize}
        \item $E[c] = c$
        \item $E[cX + Y] = cE[X] + E[Y]$
        \item $E[XY] = E[X]E[Y]$, si $X$ y $Y$ son independientes.
    \end{itemize}
\end{definicion}

